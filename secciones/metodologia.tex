\section{Metodología}
\textit{¿Qué voy a hacer?}

\subsection{Justificación de la alternativa seleccionada}

La presente investigación adopta un \textbf{enfoque híbrido} que integra dos metodologías complementarias del cálculo integral: modelado analítico mediante antiderivadas e integración numérica mediante métodos de Trapecio y Simpson. Esta combinación estratégica responde a la necesidad de balancear precisión teórica con aplicabilidad práctica.

\subsubsection*{Razón 1: Alineación con el Framework Matemático}

Nuestro framework establece una correspondencia directa entre conceptos del cálculo integral y medición energética (Tabla~\ref{tab:framework}):

\begin{itemize}
    \item Variable independiente $X$ = Tiempo $t$ (segundos) durante inferencia
    \item Variable dependiente $Y$ = Potencia instantánea $P(t)$ (watts)
    \item Función a integrar $f(x)$ = Perfil de potencia $P(t)$
    \item Resultado $Z$ = Energía total $E = \int_a^b P(t)\,dt$ (Joules)
\end{itemize}

El enfoque híbrido aprovecha esta estructura: primero modelamos analíticamente $P(t)$ como función polinómica o exponencial (Alternativa 1), luego validamos con datos discretos reales mediante integración numérica (Alternativa 2). Esto incrementa la confiabilidad de nuestras conclusiones.

\subsubsection*{Razón 2: Precisión Mediante Doble Validación}

Usando ambas metodologías logramos:

\begin{enumerate}
    \item \textbf{Modelo Analítico (Antiderivadas):} Proporciona una solución teórica exacta $E = F(b) - F(a)$ que actúa como punto de referencia cuando $P(t)$ puede aproximarse por funciones elementales.
    
    \item \textbf{Métodos Numéricos (Simpson y Trapecio):} Permiten integrar datos discretos reales de sensores de GPU/CPU sin necesidad de asumir una forma funcional específica, reduciendo el sesgo de modelado.
    
    \item \textbf{Comparación Cruzada:} Calculamos el error relativo entre ambos métodos: $\text{Error}(\%) = 100 \cdot \frac{|E_{\text{analítico}} - E_{\text{numérico}}|}{E_{\text{analítico}}}$. Errores pequeños ($< 5\%$) validan la confiabilidad de la metodología.
\end{enumerate}

\subsubsection*{Razón 3: Metodologías Numéricas Apropiadas para Datos Reales}

La integración de datos de consumo energético en centros de datos produce series temporales discretas, no funciones continuas. Por esto, Simpson y Trapecio son ideales porque:

\begin{itemize}
    \item \textbf{Trapecio:} Aproximación lineal entre puntos, rápida de implementar, bajo costo computacional.
    \item \textbf{Simpson:} Aproximación polinómica de segundo orden, mayor precisión, control de error refinando particiones.
    \item Ambos permiten análisis de convergencia ajustando el número de intervalos $n$.
\end{itemize}

\subsection{Alineación con contenidos temáticos del curso}

El curso de Cálculo Integral prescribe los siguientes contenidos temáticos, todos los cuales se integran directamente en nuestra metodología:

\begin{table}[H]
    \centering
    \caption{Mapeo entre contenidos temáticos y metodología híbrida seleccionada.}
    \label{tab:contenidos}
    \begin{tabular}{p{4cm}p{5.5cm}p{4cm}}
    \toprule
    \textbf{Contenido Temático} & \textbf{Aplicación en Metodología} & \textbf{Etapa del Proyecto} \\
    \midrule
    Antiderivada, Integral Indefinida & Modelado analítico de $P(t)$ como polinomio; cálculo de $F(t)$ & Alternativa 1 \\
    \midrule
    Métodos de integración (sustitución, partes) & Simplificación algebraica de expresiones de $P(t)$ & Alternativa 1 \\
    \midrule
    Integral Definida & Evaluación de $E = \int_a^b P(t)\,dt$ en intervalos de inferencia & Ambas \\
    \midrule
    Integral Numérica (Trapecio) & Aproximación de energía con datos discretos & Alternativa 2 \\
    \midrule
    Integral Numérica (Simpson) & Aproximación de energía con mayor precisión & Alternativa 2 \\
    \midrule
    Análisis de convergencia & Refinamiento de $n$ hasta error aceptable $< 3\%$ & Validación \\
    \bottomrule
    \end{tabular}
\end{table}

\subsection{Comparativa de métodos de integración aplicables}

\subsubsection*{Alternativa 1: Modelado Analítico (Antiderivadas)}

Cuando el perfil de potencia admite una forma funcional explícita, utilizamos antiderivadas. Un modelo típico es:

\begin{equation}
P(t) = P_0 + \alpha t + \beta t^2
\end{equation}

donde $P_0$ es potencia base, $\alpha$ y $\beta$ son coeficientes empíricos. La antiderivada es:

\begin{equation}
F(t) = P_0 t + \frac{\alpha}{2}t^2 + \frac{\beta}{3}t^3
\end{equation}

El consumo en $[a,b]$ es exacto:

\begin{equation}
E_{[a,b]} = F(b) - F(a)
\end{equation}

\textbf{Ventajas:} Exactitud, bajo costo computacional, fórmulas cerradas para análisis de sensibilidad.

\textbf{Limitaciones:} Requiere asumir una forma funcional; no captura dinámicas complejas.

\subsubsection*{Alternativa 2: Integración Numérica (Simpson y Trapecio)}

Dados $n+1$ puntos $(t_i, P_i)$ con $i = 0, 1, \ldots, n$ donde $t_0 = a$ y $t_n = b$:

\textbf{Regla del Trapecio Compuesta:}

\begin{equation}
E_{\text{Trap}} = \frac{h}{2} \left[ P_0 + 2\sum_{i=1}^{n-1} P_i + P_n \right]
\end{equation}

donde $h = \frac{b-a}{n}$ es el espaciado.

\textbf{Regla de Simpson Compuesta} (para $n$ par):

\begin{equation}
E_{\text{Simp}} = \frac{h}{3} \left[ P_0 + 4\sum_{i=1,3,5}^{n-1} P_i + 2\sum_{i=2,4,6}^{n-2} P_i + P_n \right]
\end{equation}

\textbf{Ventajas:} Aplicable a datos discretos arbitrarios, control de error, convergencia verificable.

\textbf{Limitaciones:} Aproximación, requiere datos uniformemente espaciados (o interpolación).

\subsubsection*{Validación Cruzada}

En fase de implementación, calcularemos ambas estimaciones ($E_{\text{analítico}}$ y $E_{\text{numérico}}$) sobre los mismos datos. La consistencia entre métodos (error $< 5\%$) validará la confiabilidad del análisis energético.

\subsubsection*{Alternativa 3: Integración para Comparación de Políticas}

Una vez tengamos estimaciones precisas de consumo base, aplicaremos la misma metodología numérica a escenarios alternativos (e.~g., reduced precision, batch sizes diferentes) calculando:

\begin{equation}
\Delta E = \int_a^b [P_{\text{base}}(t) - P_{\text{política}}(t)]\,dt
\end{equation}

Esto cuantifica ahorros energéticos con la misma precisión que el modelo base.
