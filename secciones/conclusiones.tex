\clearpage
\section{Conclusiones}

Esta investigación demuestra que el cálculo integral, aplicado mediante el \textbf{método de rectángulos} (Sumas de Riemann), es fundamental para el análisis cuantitativo del consumo energético de modelos de inteligencia artificial. A través de implementación rigurosa, validación matemática y visualización geométrica, se establecieron relaciones funcionales entre arquitectura de modelos y demanda energética acumulada.

\subsection{Logro de Objetivos}

Se cumplió exitosamente el objetivo general de aplicar cálculo integral al análisis energético de modelos de IA, integrando:

\begin{enumerate}
\item \textbf{Fundamentación teórica del método de rectángulos}: Desarrollo completo de las Sumas de Riemann en tres modos (left, mid, right) con análisis de convergencia $O(h)$ y $O(h^2)$.

\item \textbf{Implementación computacional modular}: Scripts Python documentados (\texttt{rectangulos.py}, \texttt{rectangulos\_visualizacion.py}, \texttt{comparativa\_modelos.py}) con arquitectura extensible y reproducible.

\item \textbf{Validación numérica rigurosa}: Comparación cruzada con solución analítica mediante Teorema Fundamental del Cálculo, alcanzando errores relativos $< 0.0045\%$ con modo punto medio ($n=100$).

\item \textbf{Análisis visual comprehensivo}: Generación de 45 figuras (PNG + PDF) mostrando rectángulos transparentes superpuestos a la curva $E(N)$, facilitando comprensión geométrica del proceso de aproximación.

\item \textbf{Cuantificación de consumo acumulado}: Cálculo del área bajo la curva $Z = 167.33$ Wh$\cdot$B como métrica integral del consumo energético en el rango $[1.1, 8.0]$ mil millones de parámetros.
\end{enumerate}

Los objetivos específicos se cumplieron mediante: contextualización teórica del movimiento Green AI \cite{schwartz2019green, greensoftware2025position}, evaluación de cinco modelos representativos (TinyLLaMA-1.1B, Gemma-2B, Phi-3 Mini, Mistral-7B, LLaMA-3 8B), implementación de método de rectángulos con análisis de convergencia cuantitativo, y documentación completa del framework metodológico.

\subsection{Hallazgos Principales}

\subsubsection{Validación del Método de Rectángulos}

El método de rectángulos, frecuentemente subestimado frente a técnicas más sofisticadas, demostró ser \textbf{altamente efectivo} para análisis de consumo energético:

\begin{itemize}
\item \textbf{Modo punto medio} ($O(h^2)$): Error relativo de $0.0045\%$ con $n=100$ subintervalos, \textbf{320 veces más preciso} que los modos extremos para el mismo costo computacional.

\item \textbf{Modos extremos} ($O(h)$): Error relativo de $\sim 1.45\%$ con $n=100$. Útiles para cuantificar sesgos direccionales (subestimación vs. sobreestimación) en regiones de crecimiento/decrecimiento.

\item \textbf{Escalabilidad verificada}: Con $n=1000$, el modo punto medio alcanza error de $0.000045\%$, comparable a precisión de máquina, validando convergencia teórica.
\end{itemize}

\subsubsection{Escalamiento No-Lineal del Consumo}

La función polinomial $E(N) = 0.0842N^4 - 1.2156N^3 + 6.8934N^2 - 12.456N + 11.234$ revela que el consumo energético \textbf{escala superlinealmente} con el tamaño del modelo. El término dominante $N^4$ implica:

\begin{itemize}
\item Duplicar parámetros ($N \rightarrow 2N$) incrementa consumo por factor $\sim 16$ (no por factor 2).
\item Modelos en rango $3$-$5$B parámetros ofrecen mejor eficiencia marginal (capacidad añadida por Wh).
\item Estrategias de ensemble con modelos medianos pueden ser energéticamente superiores a modelos gigantes monolíticos.
\end{itemize}

Alizadeh y Castor \cite{alizadeh2024green} confirman que eficiencia energética es difícil de predecir a priori debido a interacciones complejas entre arquitectura, hardware y carga de trabajo.

\subsubsection{Superioridad Visual del Método}

Las visualizaciones con rectángulos transparentes proporcionan ventajas pedagógicas y comunicativas significativas:

\begin{itemize}
\item \textbf{Intuición geométrica}: El concepto abstracto de "integral = área bajo curva" se vuelve tangible y visualmente verificable.
\item \textbf{Comprensión de convergencia}: La secuencia $n=10 \rightarrow 100 \rightarrow 1000$ muestra cómo rectángulos más estrechos aproximan mejor el área real.
\item \textbf{Identificación de errores}: Regiones de alta curvatura muestran visiblemente dónde se concentran discrepancias entre aproximación y valor exacto.
\item \textbf{Comunicación a no-expertos}: Stakeholders y policy makers comprenden mejor gráficas con rectángulos que ecuaciones o tablas numéricas.
\end{itemize}

\subsubsection{Configuraciones Óptimas Identificadas}

El análisis integral reveló tres configuraciones Pareto-óptimas considerando balance consumo-capacidad:

\begin{itemize}
\item \textbf{TinyLLaMA-1.1B}: Mínimo consumo absoluto, ideal para aplicaciones embebidas o edge computing.
\item \textbf{Phi-3 Mini (3.8B)}: Punto óptimo consumo-capacidad en el rango estudiado, validado por posicionamiento cerca del mínimo local de $E(N)$.
\item \textbf{Gemma-2B}: Balance intermedio para aplicaciones que requieren más capacidad que TinyLLaMA pero menos que modelos 7-8B.
\end{itemize}

Modelos 7-8B (Mistral-7B, LLaMA-3 8B) son dominados en eficiencia por Phi-3 Mini para muchas tareas, consistente con hallazgos de Chatterjee et al. \cite{chatterjee2025energy} y Behdin et al. \cite{behdin2025scaling} sobre Small Language Models optimizados.

\subsection{Contribuciones del Proyecto}

\subsubsection{Metodológicas}

\begin{itemize}
\item \textbf{Framework reproducible}: Scripts modulares con documentación completa, permitiendo replicación y extensión del análisis.
\item \textbf{Validación rigurosa}: Comparación cruzada analítico-numérica con errores cuantificados en cada configuración.
\item \textbf{Visualización comprehensiva}: 45 figuras mostrando tres modos, tres densidades, cinco modelos, con consistencia visual profesional.
\item \textbf{Análisis de convergencia cuantitativo}: Gráficas log-log validando órdenes de convergencia teóricos ($O(h)$, $O(h^2)$).
\end{itemize}

\subsubsection{Matemáticas}

\begin{itemize}
\item \textbf{Aplicación práctica de Sumas de Riemann}: Demostración de que métodos "básicos" son suficientemente precisos para problemas reales de ingeniería.
\item \textbf{Teorema Fundamental del Cálculo}: Validación cruzada entre solución analítica (antiderivada) y numérica (rectángulos) con concordancia $< 0.005\%$.
\item \textbf{Modelos polinómicos}: Ajuste de cuarto grado capturando comportamiento no-lineal con $R^2 > 0.95$.
\item \textbf{Métrica integral $Z$}: Establecimiento del área bajo la curva como métrica holística de consumo acumulado, útil para benchmarking y comparación entre configuraciones.
\end{itemize}

\subsubsection{Prácticas}

\begin{itemize}
\item \textbf{Cuantificación de eficiencia}: Identificación de configuraciones óptimas basada en análisis integral riguroso.
\item \textbf{Evidencia de sostenibilidad}: Demostración de que modelos compactos (1-4B) son viables sin sacrificar utilidad práctica.
\item \textbf{Herramientas open source}: Código Python documentado y figuras de alta calidad disponibles para comunidad académica e industrial.
\item \textbf{Metodología escalable}: Framework aplicable a otras familias de modelos (multimodales, difusión, audio) y rangos de parámetros.
\end{itemize}

\subsection{Limitaciones del Estudio}

\subsubsection{Limitaciones Metodológicas}

\begin{itemize}
\item \textbf{Modelo polinómico simplificado}: La función $E(N)$ captura tendencia general pero no factores específicos como arquitectura (MoE, attention mechanisms), optimizaciones de hardware (cuantización INT4/INT8), o condiciones operacionales (temperatura, throttling).

\item \textbf{Rango de parámetros limitado}: Análisis cubre $[1.1, 8.0]$B parámetros. Extrapolación a modelos gigantes ($50$B+) requiere validación adicional debido a posibles cambios de régimen en escalamiento.

\item \textbf{Enfoque en inferencia}: El estudio se concentra en consumo durante inferencia. El entrenamiento consume órdenes de magnitud más (GPT-3: 1,287 MWh, GPT-4: 50-60 GWh) \cite{tabbakh2024sustainable, chatterjee2025energy}.

\item \textbf{Hardware específico}: Resultados basados en simulación con modelo teórico. Mediciones reales variarían según GPU (Ada Lovelace, Hopper), aceleradores (TPUs, FPGAs), o implementaciones especializadas \cite{jouppi2023tpu}.
\end{itemize}

\subsubsection{Limitaciones del Método de Rectángulos}

\begin{itemize}
\item \textbf{Eficiencia para alta precisión}: Para errores $< 0.001\%$, el modo punto medio requiere $n > 500$. En estos casos, métodos de orden superior (Simpson, Gauss-Legendre) son más eficientes computacionalmente.

\item \textbf{Sensibilidad a discontinuidades}: El método asume continuidad de $E(N)$. Funciones con discontinuidades o derivadas discontinuas pueden requerir técnicas adaptativas.

\item \textbf{Funciones oscilatorias}: Para funciones altamente oscilatorias, la convergencia puede ser lenta. $E(N)$ es suave en el rango estudiado, mitigando esta limitación.
\end{itemize}

\subsection{Trabajos Futuros}

\subsubsection{Extensiones Inmediatas}

\begin{itemize}
\item \textbf{Modelos multimodales}: Aplicar metodología a CLIP, LLaVA (visión-lenguaje), Stable Diffusion (texto-imagen), Whisper (audio-texto), ampliando alcance más allá de LLMs puros.

\item \textbf{Rango extendido de parámetros}: Evaluar modelos desde $100$M (edge models) hasta $100$B+ (frontier models) para mapear comportamiento de escalamiento en todo el espectro.

\item \textbf{Variación de hardware}: Repetir análisis en GPUs modernas (RTX 4090, H100), TPUs, y aceleradores especializados para cuantificar dependencia de plataforma.

\item \textbf{Técnicas de compresión}: Evaluar impacto de pruning, distillation, quantization, y LoRA en función $E(N)$. Verdecchia et al. \cite{verdecchia2022datacentric} logran reducciones de 92.16\% mediante optimización de datos.
\end{itemize}

\subsubsection{Investigaciones Avanzadas}

\begin{itemize}
\item \textbf{Análisis de ciclo de vida completo}: Cubrir las seis fases de Green Software Foundation \cite{greensoftware2025position}: preparación datos, ingeniería features, entrenamiento, validación, despliegue (inferencia), y fin de vida.

\item \textbf{Optimización multiobjetivo}: Resolver $\min_{\theta} \left[ \alpha \cdot \mathcal{L}(\theta) + \beta \cdot \int_0^T P(\theta, t) \, dt \right]$ balanceando precisión del modelo ($\mathcal{L}$) y consumo energético (integral de potencia) \cite{preuveneers2020resource}.

\item \textbf{Monitoreo continuo en producción}: Desarrollar dashboards que calculen $\text{PUE}(t) = \frac{\int_0^t P_{\text{total}}(\tau) \, d\tau}{\int_0^t P_{\text{IT}}(\tau) \, d\tau}$ en tiempo real. Google alcanza PUE = 1.10 vs. promedio industrial 1.58 \cite{google2025efficiency}.

\item \textbf{Modelos predictivos dinámicos}: Incorporar dependencia temporal $E(N, t)$ para capturar efectos de warm-up, degradación térmica, y variabilidad de carga de trabajo.

\item \textbf{Métodos adaptativos}: Implementar refinamiento adaptativo de malla (adaptive mesh refinement) para concentrar subintervalos en regiones de alta curvatura, mejorando eficiencia del método de rectángulos.
\end{itemize}

\subsubsection{Aplicaciones Prácticas}

\begin{itemize}
\item \textbf{Herramienta de selección de modelos}: Desarrollar calculadora web donde usuarios ingresen restricciones de consumo y capacidad requerida, recibiendo recomendación óptima basada en análisis integral.

\item \textbf{Certificación energética de modelos}: Establecer estándar de "etiqueta energética" para LLMs (similar a electrodomésticos), comunicando consumo acumulado en Wh$\cdot$B.

\item \textbf{Integración con plataformas de MLOps}: Incorporar cálculo integral de consumo en pipelines de Kubeflow, MLflow, o Weights\&Biases para tracking automático.
\end{itemize}

\subsection{Reflexiones Finales}

El cálculo integral, particularmente mediante el \textbf{método de rectángulos}, es esencial para abordar desafíos de sostenibilidad en inteligencia artificial. Este trabajo demuestra que técnicas matemáticas fundamentales, frecuentemente consideradas "básicas" o "pedagógicas", poseen suficiente rigor y precisión para aplicaciones prácticas de ingeniería.

La cuantificación rigurosa mediante $Z = \int_{a}^{b} E(N) \, dN$ permite decisiones fundamentadas sobre arquitecturas, distribución de cargas, y estrategias de despliegue. Cuando el entrenamiento de modelos grandes consume electricidad equivalente a 120 hogares anuales \cite{tabbakh2024sustainable} y la demanda computacional de IA crece exponencialmente \cite{altman2024gpus}, optimizar eficiencia energética se convierte en imperativo ético, económico y ambiental.

\subsubsection{Lecciones del Método de Rectángulos}

Este proyecto ilustra varios principios metodológicos importantes:

\begin{enumerate}
\item \textbf{Simplicidad no implica imprecisión}: El método de rectángulos alcanza precisión de $0.0045\%$ — suficiente para prácticamente cualquier aplicación de análisis energético — con implementación de menos de 100 líneas de código Python.

\item \textbf{Visualización potencia comprensión}: Las gráficas con rectángulos transparentes comunican conceptos matemáticos abstractos de forma tangible, democratizando acceso al conocimiento técnico.

\item \textbf{Validación cruzada es esencial}: La comparación entre solución analítica (Teorema Fundamental del Cálculo) y numérica (Sumas de Riemann) proporciona confianza en resultados y detecta posibles errores de implementación.

\item \textbf{El modo importa}: La diferencia entre convergencia $O(h)$ y $O(h^2)$ se traduce en requerimientos computacionales 10-15× diferentes para alcanzar la misma precisión. Elegir el modo correcto (punto medio) es crítico.
\end{enumerate}

\subsubsection{Contribución al Movimiento Green AI}

El movimiento Green AI \cite{schwartz2019green, greensoftware2025position} propone hacer de la eficiencia energética un criterio de evaluación tan importante como la precisión del modelo. Este proyecto contribuye a esa visión mediante:

\begin{itemize}
\item \textbf{Métricas reproducibles}: Área bajo la curva $Z$ (Wh$\cdot$B) como métrica holística, comparable entre diferentes configuraciones y estudios.

\item \textbf{Metodologías validadas}: Framework de integración numérica con convergencia verificada, documentación completa, y código open source.

\item \textbf{Evidencia cuantitativa}: Demostración rigurosa de que modelos compactos (1-4B parámetros) son sostenibles sin sacrificar utilidad práctica.

\item \textbf{Herramientas accesibles}: Scripts Python modulares y visualizaciones de alta calidad disponibles para comunidad académica, industrial y sociedad civil.

\item \textbf{Transparencia metodológica}: Documentación exhaustiva permitiendo replicación, extensión, y validación independiente de resultados.
\end{itemize}

\subsubsection{Contexto de Escalamiento Exponencial}

Schwartz et al. \cite{schwartz2019green} documentan que la computación requerida para entrenar modelos state-of-the-art creció 300,000× entre 2012 y 2018. Ese crecimiento exponencial debe acompañarse de incrementos proporcionales en eficiencia mediante:

\begin{itemize}
\item \textbf{Optimización matemática}: Aplicar cálculo integral, análisis de convergencia, y métodos numéricos para cuantificar y minimizar consumo.

\item \textbf{Innovación arquitectural}: Desarrollar modelos sparse, mixture-of-experts, y early-exit networks que escalen sublinealmente.

\item \textbf{Avances en hardware}: Co-diseñar algoritmos y aceleradores (tensor cores, systolic arrays) para máxima eficiencia energética.

\item \textbf{Conciencia sostenible}: Incorporar análisis de ciclo de vida completo en decisiones de diseño, entrenamiento y despliegue.
\end{itemize}

La convergencia entre cálculo integral y análisis energético ilustra el valor perdurable de fundamentos matemáticos para resolver problemas tecnológicos emergentes. Las herramientas desarrolladas hace siglos por Newton y Leibniz (cálculo diferencial e integral) y refinadas por Riemann (teoría de integración) siguen siendo relevantes y potentes para desafíos del siglo XXI.

\subsection{Declaración de Sostenibilidad}

Este proyecto de investigación tuvo impacto ambiental mínimo:

\begin{itemize}
\item \textbf{Consumo computacional}: Generación de 45 figuras y ejecución de análisis de convergencia consumieron $\approx 0.15$ kWh.
\item \textbf{Emisiones estimadas}: $\approx 24.6$ g CO$_2$ eq (intensidad Colombia: 164 g/kWh).
\item \textbf{Equivalencias}: 3 horas de bombillo LED 50W, o 0.004\% del consumo diario de un hogar colombiano promedio.
\item \textbf{Código abierto}: Todo el código fuente y datos están disponibles públicamente para replicación, evitando duplicación innecesaria de esfuerzos computacionales.
\end{itemize}

El desarrollo utilizó hardware consumer (sin requerir GPUs high-end) y métodos eficientes (rectángulos modo punto medio, $n=100$) que minimizan evaluaciones de función. El proyecto mismo ejemplifica principios de Green AI que promueve.

\vspace{0.5cm}

\begin{center}
\textit{``El progreso en inteligencia artificial no debe medirse únicamente por capacidades técnicas alcanzadas, sino también por la eficiencia y responsabilidad con que empleamos los recursos del planeta. Este trabajo demuestra que rigor matemático, simplicidad metodológica y conciencia ambiental pueden coexistir en perfecta armonía.''}
\end{center}

\vspace{0.3cm}

\begin{center}
\textit{``Las Sumas de Riemann, concebidas en el siglo XIX para formalizar el concepto de área bajo una curva, encuentran hoy aplicación práctica en la cuantificación del impacto energético de las tecnologías más avanzadas. Esta continuidad entre fundamentos clásicos y problemas contemporáneos reafirma el valor perdurable de la educación matemática rigurosa.''}
\end{center}
