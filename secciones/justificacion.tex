\section{Justificación}

La optimización del consumo energético en modelos de inteligencia artificial generativa es fundamental en el contexto actual de expansión acelerada de estas tecnologías. Esta justificación se sustenta en los siguientes aspectos:

\subsection{Impacto Ambiental}

El crecimiento exponencial en la adopción de modelos de IA generativa ha generado un aumento significativo en la demanda energética de los centros de datos, incrementando la huella de carbono asociada. Según estudios recientes, el entrenamiento de modelos de gran escala produce emisiones comparables a múltiples automóviles durante toda su vida útil. La medición precisa del consumo durante la inferencia es esencial para cuantificar y mitigar este impacto ambiental en operaciones sostenibles.

\subsection{Sostenibilidad Económica}

Los costos operativos derivados del consumo energético representan una porción significativa del presupuesto de implementación y mantenimiento de sistemas de IAG. Cualquier mejora en la eficiencia energética se traduce directamente en ahorros sustanciales, aumentando la viabilidad y accesibilidad de estas tecnologías. Un caso de referencia es \textit{DeepSeek}, que ha demostrado reducciones significativas en costos operativos y consumo energético en comparación con modelos competitivos como \textit{GPT-4}, sin sacrificar rendimiento. Tales optimizaciones benefician a proveedores y a usuarios finales con tarifas más competitivas.

\subsection{Escalabilidad e Innovación Técnica}

La optimización de modelos de IA es esencial tanto para la escalabilidad como para impulsar innovación técnica. Desde la perspectiva de escalabilidad, caracterizar el consumo energético permite diseñar arquitecturas eficientes para diferentes tareas: modelos ligeros para inferencia local (como autocompletado de código) versus infraestructuras robustas para generación intensiva (generación de imágenes). Desde la innovación técnica, desarrollar modelos matemáticos precisos de medición energética abre nuevas líneas de optimización algorítmica en redes neuronales. Al priorizar eficiencia sobre la expansión de parámetros, se logran sistemas más pulidos y eficientes que impulsan avances sostenibles en el campo.

\subsection{Relevancia del Estudio}

Este estudio adquiere particular relevancia donde convergen tres factores: la demanda de servicios de IAG crece exponencialmente, aumenta la preocupación sobre sostenibilidad ambiental y eficiencia energética, y existen métricas inconsistentes para comparar eficiencia entre sistemas. Desarrollar un framework matemático riguroso basado en cálculo integral para cuantificar consumo energético en inferencia llena un vacío metodológico, permitiendo decisiones informadas sobre selección y uso de modelos de IA.