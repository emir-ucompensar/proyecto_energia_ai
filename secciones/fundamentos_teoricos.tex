\clearpage
\section{Fundamentos Teóricos}
\textit{¿Qué debo saber?}

Esta sección presenta los conceptos fundamentales del cálculo integral necesarios para comprender los métodos analíticos y numéricos aplicados al análisis energético de modelos de inteligencia artificial.

\subsection{Antiderivadas e integral indefinida}

Sea $f(x)$ una función definida en un intervalo $I$. Una función $F(x)$ es una \textbf{antiderivada} de $f(x)$ si $F^{\prime}(x) = f(x)$. La \textbf{integral indefinida} se denota como:

\begin{equation}
\int f(x) \, dx = F(x) + C
\end{equation}

donde $C$ es la constante de integración. Las propiedades fundamentales incluyen linealidad, regla de la potencia, y las integrales de funciones exponenciales y logarítmicas.

En el análisis de modelos de IA, si la potencia $P(t)$ es constante durante la inferencia, la energía total $E$ consumida en $[0, T]$ es:

\begin{equation}
E = \int_0^T P(t) \, dt = P \cdot T
\end{equation}

Para el modelo Phi-3 Mini con $P = 96$ W durante 600 segundos, obtenemos $E = 57{,}600$ J = 16 Wh, consistente con la medición experimental de 14.8 Wh (Tabla \ref{tab:metricas_experimentales}).

\subsection{Métodos de integración}

Para perfiles de potencia variables, se aplican técnicas como \textbf{integración por sustitución} ($\int f(g(x)) g^{\prime}(x) \, dx = \int f(u) \, du$) e \textbf{integración por partes} ($\int u \, dv = uv - \int v \, du$). Si la potencia durante el entrenamiento sigue un perfil exponencial $P(t) = P_{\text{max}} (1 - e^{-kt})$, la energía consumida es:

\begin{equation}
E = \int_0^T P_{\text{max}} \left(1 - e^{-kt}\right) dt = P_{\text{max}} \left(T - \frac{1 - e^{-kT}}{k}\right)
\end{equation}

\subsection{Integral definida y Teorema Fundamental del Cálculo}

La integral definida de una función continua $f$ en $[a, b]$ se define como el límite de sumas de Riemann: $\int_a^b f(x) \, dx = \lim_{n \to \infty} \sum_{i=1}^n f(x_i^*) \Delta x$. Herramientas como \texttt{nvidia-smi} y CodeCarbon \cite{luccioni2025ai} aproximan integrales continuas mediante mediciones discretas.

El \textbf{Teorema Fundamental del Cálculo} establece que si $F$ es antiderivada de $f$, entonces:

\begin{equation}
\int_a^b f(x) \, dx = F(b) - F(a)
\end{equation}

Este teorema permite calcular la energía total consumida por modelos de IA. Por ejemplo, el entrenamiento de GPT-3 requirió 1,287 MWh \cite{tabbakh2024sustainable} y GPT-4 entre 50-60 GWh \cite{chatterjee2025energy}. Las propiedades fundamentales de la integral (aditividad, linealidad, monotonía) son esenciales para el análisis energético.

\subsection{Integración numérica: Trapecio y Simpson}

Las mediciones discretas de consumo energético requieren integración numérica \cite{greensoftware2025position, schwartz2019green}. La \textbf{regla del trapecio} aproxima el área mediante trapecios con error $\mathcal{O}(h^2)$:

\begin{equation}
\int_a^b f(x) \, dx \approx \frac{h}{2} \left[f(x_0) + 2\sum_{i=1}^{n-1} f(x_i) + f(x_n)\right]
\end{equation}

La \textbf{regla de Simpson 1/3} usa parábolas con error $\mathcal{O}(h^4)$, proporcionando mayor precisión. En nuestras mediciones con \texttt{nvidia-smi}, el método del trapecio calculó 17.3 Wh para Mistral-7B versus 16.9 Wh medido (error 2.3\%).

\begin{table}[h]
\centering
\begin{tabular}{lccc}
\hline
\textbf{Método} & \textbf{Energía (Wh)} & \textbf{Error} & \textbf{Complejidad} \\
\hline
Trapecio ($n=600$) & 18.5 & 1.1\% & $\mathcal{O}(n)$ \\
Simpson 1/3 ($n=600$) & 18.31 & 0.05\% & $\mathcal{O}(n)$ \\
Medición CodeCarbon & 18.3 & --- & --- \\
\hline
\end{tabular}
\caption{Comparación de métodos de integración numérica para LLaMA-3 8B.}
\label{tab:comparacion_metodos}
\end{table}

Verdecchia et al. \cite{verdecchia2022datacentric} demostraron reducciones de 92.16\% en consumo mediante optimización de datasets, mientras que Alizadeh y Castor \cite{alizadeh2024green} confirman que la eficiencia energética es difícil de predecir entre infraestructuras.

\subsection{Aplicaciones de la integral en contexto energético}

La energía total se calcula como $E = \int_{t_0}^{t_f} P(t) \, dt$. En nuestros experimentos con 5 modelos durante 10 minutos, $E_{\text{total}} = 74.9$ Wh. La eficiencia energética se cuantifica como tokens/Wh:

\begin{equation}
\eta = \frac{N_{\text{tokens}}}{\int_0^T P(t) \, dt}
\end{equation}

TinyLLaMA alcanza 1,995 tokens/Wh versus 949 de Phi-3 Mini, demostrando que modelos compactos pueden ser más eficientes \cite{chatterjee2025energy}. La huella de carbono se calcula como $C = \int_0^T P(t) \cdot I(t) \, dt$, donde $I(t)$ es la intensidad de carbono. Nuestros tests generaron 12.3 g CO$_2$ (intensidad Colombia: 164 g/kWh).

Patterson et al. \cite{patterson2022carbon} demuestran reducciones de 100x en energía y 1000x en emisiones mediante mejores prácticas \cite{greensoftware2025position}. La optimización multi-objetivo considera trade-offs entre precisión y consumo mediante funciones de utilidad que maximizan rendimiento mientras minimizan energía.

Behdin et al. \cite{behdin2025scaling} muestran que Small Language Models con compresión estructurada logran calidad comparable a modelos grandes con costos reducidos. La eficiencia en datacenters se mide mediante PUE = $\frac{\int_0^T P_{\text{DC}}(t) \, dt}{\int_0^T P_{\text{IT}}(t) \, dt}$, donde Google reporta 1.10 versus promedio industrial de 1.58 \cite{google2025efficiency}.

\begin{table}[h]
\centering
\begin{tabular}{lcccccc}
\hline
\textbf{Modelo} & \textbf{Tamaño} & \textbf{Tokens/s} & \textbf{Potencia} & \textbf{Energía} & \textbf{Eficiencia} & \textbf{VRAM} \\
 & \textbf{(B)} & & \textbf{GPU (W)} & \textbf{(Wh)} & \textbf{(t/Wh)} & \textbf{(GiB)} \\
\hline
Phi-3 Mini & 3.8 & 23.4 & 96 & 14.8 & 949 & 4.9 \\
LLaMA-3 8B & 8.0 & 17.1 & 110 & 18.3 & 560 & 5.6 \\
Mistral-7B & 7.0 & 19.8 & 104 & 16.9 & 703 & 5.3 \\
Gemma-2B & 2.0 & 31.2 & 88 & 13.2 & 1,418 & 4.1 \\
TinyLLaMA & 1.1 & 38.9 & 82 & 11.7 & 1,995 & 3.8 \\
\hline
\end{tabular}
\caption{Métricas experimentales de 5 modelos LLM (test 10 min). Hardware: GTX 1660 Ti, Ryzen 7 5800X. Herramientas: nvidia-smi, CodeCarbon.}
\label{tab:metricas_experimentales}
\end{table}

\subsection{Síntesis}

El cálculo integral proporciona las herramientas para cuantificar consumo energético mediante integrales definidas, aproximar mediciones discretas con métodos numéricos, optimizar diseños considerando trade-offs, y comparar arquitecturas mediante métricas derivadas de integrales. La convergencia entre teoría matemática y mediciones prácticas con herramientas como CodeCarbon permite avanzar hacia una IA sostenible \cite{tabbakh2024sustainable}, donde la eficiencia energética es criterio fundamental junto con precisión y rendimiento.

