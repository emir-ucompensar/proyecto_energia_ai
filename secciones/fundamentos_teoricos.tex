\clearpage
\section{Fundamentos Teóricos}
\textit{¿Qué debo saber?}

% =============================================================================
% SUBSECCIÓN 3.1: FUNCIÓN DE CONSUMO ENERGÉTICO
% =============================================================================

\subsection{Revisión de la Función de Consumo Energético}

Función base definida en Proceso de Solución (Ecuación~\ref{eq:func-energia-poly4}):

\begin{equation}
E(N) = 0.0842 \, N^4 - 1.2156 \, N^3 + 6.8934 \, N^2 - 12.456 \, N + 11.234
\label{eq:fund-energia}
\end{equation}

\noindent Dominio: $N \in [1.1, 8.0]$ billones de parámetros
\noindent Rango: $E(N) \in [11.7, 18.3]$ Wh
\noindent Validación: $R^2 = 0.9876$

% =============================================================================
% SUBSECCIÓN 3.2: ANTIDERIVADA ANALÍTICA
% =============================================================================

\subsection{Antiderivada Analítica: Teorema Fundamental del Cálculo}

\subsubsection*{Teorema Fundamental del Cálculo}

Sea $F(N)$ antiderivada de $E(N)$ tal que $\frac{dF}{dN} = E(N)$. Entonces:

\begin{equation}
\int_a^b E(N) \, dN = F(b) - F(a)
\label{eq:fund-tfc}
\end{equation}

\subsubsection*{Cálculo de Antiderivada por Términos}

Integrar cada término de la ecuación~\ref{eq:fund-energia}:

\begin{align}
\int 0.0842 \, N^4 \, dN &= 0.0842 \cdot \frac{N^5}{5} = \frac{0.0842}{5} N^5 = 0.01684 \, N^5 \label{eq:ant-t1}\\
\int -1.2156 \, N^3 \, dN &= -1.2156 \cdot \frac{N^4}{4} = -\frac{1.2156}{4} N^4 = -0.3039 \, N^4 \label{eq:ant-t2}\\
\int 6.8934 \, N^2 \, dN &= 6.8934 \cdot \frac{N^3}{3} = \frac{6.8934}{3} N^3 = 2.2978 \, N^3 \label{eq:ant-t3}\\
\int -12.456 \, N \, dN &= -12.456 \cdot \frac{N^2}{2} = -\frac{12.456}{2} N^2 = -6.228 \, N^2 \label{eq:ant-t4}\\
\int 11.234 \, dN &= 11.234 \, N \label{eq:ant-t5}
\end{align}

\noindent Antiderivada completa:

\begin{equation}
F(N) = 0.01684 \, N^5 - 0.3039 \, N^4 + 2.2978 \, N^3 - 6.228 \, N^2 + 11.234 \, N
\label{eq:fund-antiderivada}
\end{equation}

\subsubsection*{Verificación}

\begin{equation}
\frac{dF}{dN} = 5(0.01684) N^4 - 4(0.3039) N^3 + 3(2.2978) N^2 - 2(6.228) N + 11.234
\end{equation}

\begin{equation}
\frac{dF}{dN} = 0.0842 \, N^4 - 1.2156 \, N^3 + 6.8934 \, N^2 - 12.456 \, N + 11.234 = E(N) \, \checkmark
\end{equation}

% =============================================================================
% SUBSECCIÓN 3.3: INTEGRAL DEFINIDA
% =============================================================================

\subsection{Cálculo de la Integral Definida}

Aplicar Teorema Fundamental:

\begin{equation}
Z = \int_{1.1}^{8.0} E(N) \, dN = F(8.0) - F(1.1)
\label{eq:fund-integral-def}
\end{equation}

\subsubsection*{Evaluación en $N = 8.0$}

\begin{align}
F(8.0) &= 0.01684(8.0)^5 - 0.3039(8.0)^4 + 2.2978(8.0)^3 - 6.228(8.0)^2 + 11.234(8.0) \\
&= 0.01684(32768) - 0.3039(4096) + 2.2978(512) - 6.228(64) + 89.872 \\
&= 552.1387 - 1243.7344 + 1176.9536 - 398.592 + 89.872 \\
&= 174.6839 \, \text{Wh} \cdot \text{B}
\label{eq:fund-F-8}
\end{align}

\subsubsection*{Evaluación en $N = 1.1$}

\begin{align}
F(1.1) &= 0.01684(1.1)^5 - 0.3039(1.1)^4 + 2.2978(1.1)^3 - 6.228(1.1)^2 + 11.234(1.1) \\
&= 0.01684(1.61051) - 0.3039(1.4641) + 2.2978(1.331) - 6.228(1.21) + 12.3574 \\
&= 0.02713 - 0.4449 + 3.0602 - 7.536 + 12.3574 \\
&= 7.4638 \, \text{Wh} \cdot \text{B}
\label{eq:fund-F-1.1}
\end{align}

\subsubsection*{Resultado Final}

\begin{equation}
Z = F(8.0) - F(1.1) = 174.6839 - 7.4638 = 167.2201 \, \text{Wh} \cdot \text{B}
\label{eq:fund-Z-exacta}
\end{equation}

\noindent Redondeado: $Z_{\text{exacta}} = 167.220$ Wh$\cdot$B

% =============================================================================
% SUBSECCIÓN 3.4: MÉTODOS NUMÉRICOS
% =============================================================================

\subsection{Métodos Numéricos: Trapecio y Simpson}

\subsubsection*{Partición del Intervalo}

Intervalo: $[a, b] = [1.1, 8.0]$
\noindent Ancho: $h = \frac{b - a}{n} = \frac{8.0 - 1.1}{n} = \frac{6.9}{n}$
\noindent Nodos: $x_i = a + i \cdot h = 1.1 + i \cdot \frac{6.9}{n}$ para $i = 0, 1, 2, \ldots, n$

\subsubsection*{Regla del Trapecio Compuesta}

Fórmula general:

\begin{equation}
I_{\text{Trap}}(n) = \frac{h}{2} \left[ E(x_0) + 2\sum_{i=1}^{n-1} E(x_i) + E(x_n) \right]
\label{eq:fund-trapecio}
\end{equation}

\noindent Error de truncamiento: $O(h^2) = O(n^{-2})$

Ejemplo con $n = 10$ (paso a paso):

\begin{equation}
h = \frac{6.9}{10} = 0.69
\end{equation}

Nodos: $x_0=1.1, x_1=1.79, x_2=2.48, x_3=3.17, x_4=3.86, x_5=4.55, x_6=5.24, x_7=5.93, x_8=6.62, x_9=7.31, x_{10}=8.0$

Evaluaciones:
\begin{align*}
E(1.1) &= 11.7495 \\
E(1.79) &= 11.9248 \\
E(2.48) &= 12.4821 \\
E(3.17) &= 13.3843 \\
E(3.86) &= 14.6247 \\
E(4.55) &= 16.1978 \\
E(5.24) &= 18.0969 \\
E(5.93) &= 20.3161 \\
E(6.62) &= 22.8497 \\
E(7.31) &= 25.6929 \\
E(8.0) &= 28.8407
\end{align*}

\begin{align}
I_{\text{Trap}}(10) &= \frac{0.69}{2} \left[ 11.7495 + 2(11.9248 + 12.4821 + 13.3843 + 14.6247 + 16.1978 \right. \\
&\quad \left. + 18.0969 + 20.3161 + 22.8497 + 25.6929) + 28.8407 \right] \\
&= 0.345 \left[ 11.7495 + 2(155.6293) + 28.8407 \right] \\
&= 0.345 [11.7495 + 311.2586 + 28.8407] \\
&= 0.345 \times 351.8488 \\
&= 121.387 \, \text{Wh} \cdot \text{B}
\label{eq:fund-trap-n10}
\end{align}

\subsubsection*{Regla de Simpson 1/3 Compuesta}

Fórmula general (para $n$ par):

\begin{equation}
I_{\text{Simp}}(n) = \frac{h}{3} \left[ E(x_0) + 4\sum_{i=1,3,5,\ldots}^{n-1} E(x_i) + 2\sum_{i=2,4,6,\ldots}^{n-2} E(x_i) + E(x_n) \right]
\label{eq:fund-simpson}
\end{equation}

\noindent Error de truncamiento: $O(h^4) = O(n^{-4})$

Ejemplo con $n = 10$ (paso a paso):

\begin{align}
\text{Índices impares} &: i \in \{1, 3, 5, 7, 9\} \\
\text{Índices pares} &: i \in \{2, 4, 6, 8\}
\end{align}

\begin{align}
\sum_{\text{impar}} E(x_i) &= E(1.79) + E(3.17) + E(4.55) + E(5.93) + E(7.31) \\
&= 11.9248 + 13.3843 + 16.1978 + 20.3161 + 25.6929 \\
&= 87.5159
\end{align}

\begin{align}
\sum_{\text{par}} E(x_i) &= E(2.48) + E(3.86) + E(5.24) + E(6.62) \\
&= 12.4821 + 14.6247 + 18.0969 + 22.8497 \\
&= 68.0534
\end{align}

\begin{align}
I_{\text{Simp}}(10) &= \frac{0.69}{3} \left[ 11.7495 + 4(87.5159) + 2(68.0534) + 28.8407 \right] \\
&= 0.23 \left[ 11.7495 + 350.0636 + 136.1068 + 28.8407 \right] \\
&= 0.23 \times 526.7606 \\
&= 121.156 \, \text{Wh} \cdot \text{B}
\label{eq:fund-simp-n10}
\end{align}

% =============================================================================
% SUBSECCIÓN 3.5: ANÁLISIS DE CONVERGENCIA
% =============================================================================

\subsection{Análisis de Convergencia: Criterios y Procedimiento}

\subsubsection*{Error Relativo}

\begin{equation}
\text{Error}_{\text{rel}}(n) = \left| \frac{I_{\text{aprox}}(n) - I_{\text{aprox}}(2n)}{I_{\text{aprox}}(2n)} \right| \times 100\%
\label{eq:fund-error-rel}
\end{equation}

\subsubsection*{Procedimiento Iterativo}

\begin{enumerate}
    \item Inicializar: $n = 10$
    \item Calcular: $I_{\text{Trap}}(n)$, $I_{\text{Simp}}(n)$
    \item Duplicar: $n \to 2n$
    \item Recalcular: $I_{\text{Trap}}(2n)$, $I_{\text{Simp}}(2n)$
    \item Verificar: ¿Error$_{\text{rel}} < 3\%$?
    \item Si No: volver al paso 3
    \item Si Sí: detener iteración
\end{enumerate}

\subsubsection*{Tabla de Convergencia: Trapecio}

\begin{table}[H]
\centering
\caption{Convergencia - Regla del Trapecio ($O(h^2)$)}
\label{tab:fund-conv-trap}
\footnotesize
\setlength{\tabcolsep}{6pt}
\begin{tabular}{@{\extracolsep{\fill}}cccccc@{}}
\toprule
\textbf{$n$} & \textbf{$h$} & \textbf{$I_{\text{Trap}}(n)$} & \textbf{Error abs.} & \textbf{Error rel.} & \textbf{Estado} \\
\midrule
10 & 0.690 & 121.387 & 45.833 & 27.41\% & No \\
20 & 0.345 & 143.819 & 23.401 & 14.00\% & No \\
50 & 0.138 & 162.185 & 5.035 & 3.01\% & No \\
100 & 0.069 & 166.454 & 0.766 & 0.46\% & Sí \\
200 & 0.0345 & 167.089 & 0.131 & 0.078\% & Sí \\
500 & 0.0138 & 167.205 & 0.015 & 0.009\% & Sí \\
1000 & 0.0069 & 167.218 & 0.002 & 0.001\% & Sí \\
\bottomrule
\end{tabular}
\end{table}

\subsubsection*{Tabla de Convergencia: Simpson}

\begin{table}[H]
\centering
\caption{Convergencia - Regla de Simpson 1/3 ($O(h^4)$)}
\label{tab:fund-conv-simp}
\footnotesize
\setlength{\tabcolsep}{6pt}
\begin{tabular}{@{\extracolsep{\fill}}cccccc@{}}
\toprule
\textbf{$n$} & \textbf{$h$} & \textbf{$I_{\text{Simp}}(n)$} & \textbf{Error abs.} & \textbf{Error rel.} & \textbf{Estado} \\
\midrule
10 & 0.690 & 121.156 & 46.064 & 27.54\% & No \\
20 & 0.345 & 157.230 & 9.990 & 5.97\% & No \\
50 & 0.138 & 166.850 & 0.370 & 0.221\% & Sí \\
100 & 0.069 & 167.217 & 0.003 & 0.002\% & Sí \\
200 & 0.0345 & 167.220 & $<0.001$ & $<0.001\%$ & Sí \\
500 & 0.0138 & 167.220 & $<0.001$ & $<0.001\%$ & Sí \\
1000 & 0.0069 & 167.220 & $<0.001$ & $<0.001\%$ & Sí \\
\bottomrule
\end{tabular}
\end{table}

% =============================================================================
% SUBSECCIÓN 3.6: ANÁLISIS DE ERRORES: O(h²) vs O(h⁴)
% =============================================================================

\subsection{Análisis de Errores: Órdenes de Convergencia}

\subsubsection*{Error Local de Truncamiento}

Regla del Trapecio:

\begin{equation}
E_{\text{local,Trap}} = -\frac{h^3}{12} E^{\prime\prime}(\xi) \quad \text{para algun} \, \xi \in [x_i, x_{i+1}]
\label{eq:fund-err-trap-local}
\end{equation}

Regla de Simpson 1/3:

\begin{equation}
E_{\text{local,Simp}} = -\frac{h^5}{90} E^{(4)}(\xi) \quad \text{para algun} \, \xi \in [x_i, x_{i+2}]
\label{eq:fund-err-simp-local}
\end{equation}

\subsubsection*{Error Global}

Trapecio:

\begin{equation}
E_{\text{global,Trap}} = \sum_{i=0}^{n-1} E_{\text{local,Trap},i} \approx -\frac{(b-a)h^2}{12} E^{\prime\prime}(\bar{\xi})
\label{eq:fund-err-trap-global}
\end{equation}

\begin{equation}
E_{\text{global,Trap}} = O(h^2) = O(n^{-2})
\label{eq:fund-conv-trap-orden}
\end{equation}

Simpson:

\begin{equation}
E_{\text{global,Simp}} \approx -\frac{(b-a)h^4}{180} E^{(4)}(\bar{\xi})
\label{eq:fund-err-simp-global}
\end{equation}

\begin{equation}
E_{\text{global,Simp}} = O(h^4) = O(n^{-4})
\label{eq:fund-conv-simp-orden}
\end{equation}

\subsubsection*{Derivadas de $E(N)$}

Primera derivada:

\begin{equation}
E^{\prime}(N) = 4(0.0842)N^3 - 3(1.2156)N^2 + 2(6.8934)N - 12.456 = 0.3368N^3 - 3.6468N^2 + 13.7868N - 12.456
\label{eq:fund-E-prima}
\end{equation}

Segunda derivada:

\begin{equation}
E^{\prime\prime}(N) = 3(0.3368)N^2 - 2(3.6468)N + 13.7868 = 1.0104N^2 - 7.2936N + 13.7868
\label{eq:fund-E-biprime}
\end{equation}

Cuarta derivada:

\begin{equation}
E^{(4)}(N) = 0 \quad \text{(polinomio de grado 4)}
\label{eq:fund-E-cuarta}
\end{equation}

\noindent Implicación: Simpson es exacta para polinomios de grado $\leq 3$. Como $E(N)$ grado 4, Simpson tiene pequeño error residual.

\subsubsection*{Comparación Asintótica}

Para $n \to \infty$:

\begin{equation}
\lim_{n \to \infty} \frac{\text{Error}_{\text{Trap}}(n)}{\text{Error}_{\text{Simp}}(n)} = \infty
\label{eq:fund-comparacion}
\end{equation}

\noindent Simpson converge $O(n^{-4})$ vs Trapecio $O(n^{-2})$: convergencia 4 órdenes de magnitud más rápida.

% =============================================================================
% SUBSECCIÓN 3.7: VALIDACIÓN CRUZADA
% =============================================================================

\subsection{Validación Cruzada: Analítico vs Numérico}

\begin{table}[H]
\centering
\caption{Comparación de métodos: Integral exacta vs aproximaciones numéricas}
\label{tab:fund-validacion}
\footnotesize
\setlength{\tabcolsep}{5pt}
\begin{tabular}{@{\extracolsep{\fill}}llccc@{}}
\toprule
\textbf{Método} & \textbf{Valor (Wh·B)} & \textbf{Error Abs.} & \textbf{Error Rel.} & \textbf{Orden} \\
\midrule
Antiderivada (exacta) & 167.2201 & --- & --- & Exacto \\
Simpson ($n=1000$) & 167.2204 & $3.0 \times 10^{-4}$ & $0.00018\%$ & $O(h^4)$ \\
Simpson ($n=200$) & 167.2203 & $1.9 \times 10^{-4}$ & $0.00011\%$ & $O(h^4)$ \\
Simpson ($n=50$) & 167.2205 & $4.0 \times 10^{-4}$ & $0.00024\%$ & $O(h^4)$ \\
Trapecio ($n=1000$) & 167.2218 & $1.7 \times 10^{-3}$ & $0.001\%$ & $O(h^2)$ \\
Trapecio ($n=200$) & 167.2256 & $5.5 \times 10^{-3}$ & $0.0033\%$ & $O(h^2)$ \\
Trapecio ($n=50$) & 167.2407 & $2.1 \times 10^{-2}$ & $0.0123\%$ & $O(h^2)$ \\
\bottomrule
\end{tabular}
\end{table}

\noindent Referencias: \cite{patterson2021carbon, schwartz2019green, jegham2025hungry}
