\section{Introducción}

La adopción masiva de modelos de inteligencia artificial generativa (LLMs) ha impulsado un crecimiento acelerado en la demanda energética de los centros de datos. Estos modelos requieren grandes volúmenes de potencia de computo tanto en la fase de entrenamiento como en la etapa de inferencia o uso, lo que genera preocupaciones por el consumo eléctrico, el impacto ambiental y los costos de operación asociados a estas optimizaciones.

El consumo energético de la inteligencia artificial se distribuye en dos componentes principales: (a) la energía requerida para entrenar modelos de gran escala, y (b) la energía demandada en la operación continua de inferencia. Este último componente —la fase de producción— resulta especialmente relevante debido a su carácter constante y global.

Debido al exito de los modelos de inteligencia artificial generativa hemos visto cómo no sólo nuestras vidas se han visto rodeadas de éstas tecnologías, siendo el modelo de lenguaje más popular \textit{ChatGPT}.
Esto tiene impactos interesantes, porque muchas personas que suelen ser vistas como el usuario promedio únicamente usan \textit{ChatGPT}, porque es el único modelo de lenguaje que conocen. Muchas veces estos mismos usuarios son los que más impacto tienen, porque no saben cómo funcionan estos modelos y se pasan todo el día charlando con ellos o generando imágenes sin ningún control. 

Por la contra parte tenemos a los usuarios más avanzados y tecnológicos que saben muy bien cómo estas redes neuronales operan, y son conscientes del impacto que tienen sus peticiones, y por ende hacen un uso más racional de estas tecnologías.

Conviene hacer una aclaración para concluir esta introducción: pese a las preocupaciones sobre su huella energética, las tecnologías digitales —incluida la inteligencia artificial— siguen siendo más limpias que muchas actividades industriales tradicionales.

Sectores como el automotriz, el alimentario o el de la moda generan impactos ambientales mucho más visibles y persistentes, desde emisiones de CO$_2$ hasta contaminación por desechos o procesos químicos. En este documento me centraré en comprender los costos energéticos de la IA y propondré nuevas maneras sencillas de calcular los costos e impactos, con el propósito de dimensionar de forma más justa su verdadero impacto.
